\PassOptionsToPackage{unicode=true}{hyperref} % options for packages loaded elsewhere
\PassOptionsToPackage{hyphens}{url}
%
\documentclass[]{article}
\usepackage{lmodern}
\usepackage{amssymb,amsmath}
\usepackage{ifxetex,ifluatex}
\usepackage{fixltx2e} % provides \textsubscript
\ifnum 0\ifxetex 1\fi\ifluatex 1\fi=0 % if pdftex
  \usepackage[T1]{fontenc}
  \usepackage[utf8]{inputenc}
  \usepackage{textcomp} % provides euro and other symbols
\else % if luatex or xelatex
  \usepackage{unicode-math}
  \defaultfontfeatures{Ligatures=TeX,Scale=MatchLowercase}
\fi
% use upquote if available, for straight quotes in verbatim environments
\IfFileExists{upquote.sty}{\usepackage{upquote}}{}
% use microtype if available
\IfFileExists{microtype.sty}{%
\usepackage[]{microtype}
\UseMicrotypeSet[protrusion]{basicmath} % disable protrusion for tt fonts
}{}
\IfFileExists{parskip.sty}{%
\usepackage{parskip}
}{% else
\setlength{\parindent}{0pt}
\setlength{\parskip}{6pt plus 2pt minus 1pt}
}
\usepackage{hyperref}
\hypersetup{
            pdfborder={0 0 0},
            breaklinks=true}
\urlstyle{same}  % don't use monospace font for urls
\setlength{\emergencystretch}{3em}  % prevent overfull lines
\providecommand{\tightlist}{%
  \setlength{\itemsep}{0pt}\setlength{\parskip}{0pt}}
\setcounter{secnumdepth}{0}
% Redefines (sub)paragraphs to behave more like sections
\ifx\paragraph\undefined\else
\let\oldparagraph\paragraph
\renewcommand{\paragraph}[1]{\oldparagraph{#1}\mbox{}}
\fi
\ifx\subparagraph\undefined\else
\let\oldsubparagraph\subparagraph
\renewcommand{\subparagraph}[1]{\oldsubparagraph{#1}\mbox{}}
\fi

% set default figure placement to htbp
\makeatletter
\def\fps@figure{htbp}
\makeatother


\date{}

\begin{document}

\hypertarget{Upux20atux20night:ux20Miriamux20onux20theux20mind}{%
\section{Up at night: Miriam on the
mind}\label{Upux20atux20night:ux20Miriamux20onux20theux20mind}}

It was never my intention to cut off contact with you. I am sorry for having hurt you. It was never my intention to do that.

Putting it that serves as a reminder that part of what I tried to say in our last conversation was that "in spite of our intentions, often a different result occurs than what we intended." I kept repeating "unintended consequences." We often end up hurting others even when it is not our intention. I speak of and for myself.

I have experienced a deep sense of family -\/- what it means to be father, daughter, brother, sister -\/- with Mary Pat's family. I think it may be similar to what my brother David experienced with his in-laws.  Maybe Bill and Shirley also. Sadly, -\/- and it may well be because of mistakes that I have made -\/- my family has not been able to provide the same sense of family.

My friends have said to me things like, "I can't imagine doing that ...  as a (parent, brother, son)." I am not totally off my rocker for expecting certain things of "family." But an ordinary family is not what I have. I have a broken one. Of course one of my assumptions for many years is that most, if not all, people come from broken families. But I observe that some take \protect\hypertarget{alienation}{}{}\textbf{alienation} to be an expected outcome in a family. Others do not. I just have to accept that my family is what it is.

I have friends who have been heroic in their actions as (parent, brother, son). I wish I could be more open about that, but their story is not my story to tell.

In rather bitter contrast I have seen members of families in actions that are abominable, venal, dishonest, disrespectful -\/- I could go on.  It makes a difference what we do and how we act. But I suppose that you know that; so forgive me for sounding like I am lecturing. I don't know how to respond to people who look like they are members of my family
when they act like that.

When we talked I was feeling especially hurt at what feels like other people stealing my children from me. Anger at what other people have done to hurt my children. I have no right to expect that you would in fact be able to empathize with my pain, but I do know of your deep devotion and love for your own children. I am so grateful for that devotion. I forget that I am not just anybody off the street to you, but a father who was divorced from your mother when you were but quite
young.

Now after the funeral of my mother I have observed members of my family acting in the most dishonest, disrespectful way, but with a smile as if they were saints at work. I don't know what to do with that.

Sometimes I just wish that I could be a father with a daughter the way I see others enjoying it. I actually really love being a grandfather. I forget that all of my children have issues with me and that I will never have such a relationship. It makes me sad.

\textgreater{} I feel your pain" is not a good response to someone who has lost a child.

\end{document}
