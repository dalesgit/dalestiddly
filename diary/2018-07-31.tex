\PassOptionsToPackage{unicode=true}{hyperref} % options for packages loaded elsewhere
\PassOptionsToPackage{hyphens}{url}
%
\documentclass[]{article}
\usepackage{lmodern}
\usepackage{amssymb,amsmath}
\usepackage{ifxetex,ifluatex}
\usepackage{fixltx2e} % provides \textsubscript
\ifnum 0\ifxetex 1\fi\ifluatex 1\fi=0 % if pdftex
  \usepackage[T1]{fontenc}
  \usepackage[utf8]{inputenc}
  \usepackage{textcomp} % provides euro and other symbols
\else % if luatex or xelatex
  \usepackage{unicode-math}
  \defaultfontfeatures{Ligatures=TeX,Scale=MatchLowercase}
\fi
% use upquote if available, for straight quotes in verbatim environments
\IfFileExists{upquote.sty}{\usepackage{upquote}}{}
% use microtype if available
\IfFileExists{microtype.sty}{%
\usepackage[]{microtype}
\UseMicrotypeSet[protrusion]{basicmath} % disable protrusion for tt fonts
}{}
\IfFileExists{parskip.sty}{%
\usepackage{parskip}
}{% else
\setlength{\parindent}{0pt}
\setlength{\parskip}{6pt plus 2pt minus 1pt}
}
\usepackage{hyperref}
\hypersetup{
            pdfborder={0 0 0},
            breaklinks=true}
\urlstyle{same}  % don't use monospace font for urls
\setlength{\emergencystretch}{3em}  % prevent overfull lines
\providecommand{\tightlist}{%
  \setlength{\itemsep}{0pt}\setlength{\parskip}{0pt}}
\setcounter{secnumdepth}{0}
% Redefines (sub)paragraphs to behave more like sections
\ifx\paragraph\undefined\else
\let\oldparagraph\paragraph
\renewcommand{\paragraph}[1]{\oldparagraph{#1}\mbox{}}
\fi
\ifx\subparagraph\undefined\else
\let\oldsubparagraph\subparagraph
\renewcommand{\subparagraph}[1]{\oldsubparagraph{#1}\mbox{}}
\fi

% set default figure placement to htbp
\makeatletter
\def\fps@figure{htbp}
\makeatother


\date{}

\begin{document}

\hypertarget{Sourceux20ofux20anxietyux3f}{%
\section{Source of anxiety?}\label{Sourceux20ofux20anxietyux3f}}

One part of me thinks that the anxiety I felt the other night was
related to the upcoming
*\protect\hypertarget{retirement}{}{}\textbf{retirement}* that will
leave me with little incentive to develop my mind and curiosity. Compare
the way my music inventiveness went the wayside after leaving parish
ministry. Or compare the way I gave up a whole range of publication
interest. Or at Chaminade my developing interest in Catholic thought of
religious pedagogy.

Read yesterday about the past convention of AAR and a concern that the
teaching of religion be taught with an awareness of the most recent
*\protect\hypertarget{learningux20theories}{}{}\textbf{learning
theories}*. It seemed so bogus and scientifically jargonny to me. It
wasn't that that I was interested in. But maybe I have something to
learn there?

\hypertarget{Draftux202ux20forux20Miriam}{%
\section{Draft 2 for Miriam}\label{Draftux202ux20forux20Miriam}}

It was never my intention to cut off contact with you. I am sorry for
having hurt you. It was never my intention to do that.

Putting it that way serves as a reminder to me that part of what I tried
to say in our last conversation was that "in spite of good intentions,
often a different result occurs from what we intended." I kept repeating
"unintended consequences." What I meant was that our intentions and the
results of our decisions and actions don't necessarily line up. We often
end up hurting others even when it is not our intention. I speak of and
for myself.

I have experienced a deep sense of family -\/- what it means to be
father, daughter, brother, sister -\/- with Mary Pat's family. I think
it may be similar to what my brother David experienced with his in-laws.
Maybe Bill and Shirley with their in-laws also. It might well be true
for you. I don't know. Sadly, -\/- and it may well be because of my
mistakes -\/- my family has not been able to provide the same sense of
family.

When talking about my family, some of my friends have said to me things
like, "I can't imagine doing that ... as a (parent, brother, son)." I am
not totally off my rocker for expecting certain things of "family." But
an ordinary family is not what I have. I have a broken one. Of course
one of my assumptions for many years is that most, if not all, people
come from broken families. I observe that some take
\protect\hypertarget{alienation}{}{}\textbf{alienation} to be an
expected outcome in a family. Others do not. I just have to accept that
my family is what it is.

Some of my friends have been heroic in their actions as (parent,
brother, son). I admire them so much. I wish I could be more open with
you about that, but their story is not my story to tell. In rather
bitter contrast I have seen members of my family -\/-
\protect\hypertarget{notux20youux20beux20assured}{}{}\textbf{not you be
assured} -\/- in actions that are abominable, venal, dishonest,
disrespectful -\/- I could go on. It makes a difference what we do and
how we act. I am quite certain the family member I'm talking about would
have wonderful things to say about his/her intentions. Forgive me if it
sounds like I'm ranting or lecturing. I'm not intending that. I want to
be vulnerable with you and honest at the same time.

When we talked I was feeling especially hurt at what feels like other
people stealing my children from me. Anger at what other people have
done to hurt my children. I have no right to expect that you would in
fact be able to empathize with my pain, but I do know of the deep
devotion and love you have for your own children. I am grateful for that
devotion. I forget that I am not just anybody off the street to you, but
a father who was divorced from your mother when you were but quite
young.

Sometimes I just wish that I could be a father with a daughter the way I
see others enjoying it. I actually really love being a grandfather. I
forget that all of my children have issues with me and that I will never
have such a relationship. It makes me sad.

I do not want there to be broken relations with you and your family. I
apologize for the actions I have done that have contributed to the
brokenness. Please forgive me.

Love, Dad

\end{document}
